\section{Introduction}

\begin{frame}{Motivations}
    \begin{itemize}
        \item Venant de la communauté mécanique, les concepts de fibrés, connexions et dérivées covariantes sont difficiles à appréhender ;
        \item Les références en la matière (\alert{\cite{bleecker_gauge_1981}} ; \cite{kobayashi_foundations_1996}) sont difficiles à lire ;
        \item Il est difficile de trouver des points de vues simplificateurs et accessibles sur ce sujet ;
        \item Le sujet va être largement abordé durant le GDR :
    \end{itemize}
\end{frame}

\begin{RecapFrame}{Contexte}
    \begin{block}{Usages}
        \begin{itemize}
            \item On confond souvent la notion de dérivée covariante avec celle de connexion (ex : page wikipédia sur la connexion de Levi-Civita) alors que le lien n'est pas si trivial (à mon sens) ;
            \item \alert{Les connexions dans un sens plus général} seront abordées dans plusieurs exposés ;
            \item Le fibré des repères est un prototype de fibré principal qui est la structure de base des théories de jauges.
            \item En particulier en électro-magnétisme (\cite{weyl_gravitation_1918}) ;
        \end{itemize}
    \end{block}
    \begin{alertblock}{title}
        \begin{equation*}
            R_g^*\omega = g^{-1}\omega g.
        \end{equation*}
    \end{alertblock}
\end{RecapFrame}

\begin{frame}{Sommaire}
    \begin{columns}
        \begin{column}{0.4\textwidth}
            \tableofcontents
        \end{column}
        \begin{column}{0.5\textwidth}
            \begin{block}{Objectifs de la présentation}
                On va définir un objet, une \alert{connexion $\omega$}, sur le \alert{fibré des repères $\LM$, ensemble des repères en chaque points d'une variété $\MM$}, et on va voir qu'elle s'identifie à une \alert{dérivée covariante $\nabla$ sur $T\MM$} (et un peu plus).
            \end{block}
        \end{column}
    \end{columns}
\end{frame}